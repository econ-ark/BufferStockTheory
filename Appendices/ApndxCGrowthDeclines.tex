% -*- mode: LaTeX; TeX-PDF-mode: t; -*-
% LaTeX path to the root directory of the current project, from the directory in which this file resides
% and path to econtexPaths which defines the rest of the paths like \FigDir
\providecommand{\econtexRoot}{}\renewcommand{\econtexRoot}{.}
\providecommand{\econtexPaths}{}\renewcommand{\econtexPaths}{\econtexRoot/Resources/econtexPaths}
% The \commands below are required to allow sharing of the same base code via Github between TeXLive on a local machine and Overleaf (which is a proxy for "a standard distribution of LaTeX").  This is an ugly solution to the requirement that custom LaTeX packages be accessible, and that Overleaf prohibits symbolic links

\providecommand{\econtex}{\econtexRoot/Resources/texmf-local/tex/latex/econtex}
\providecommand{\pdfsuppressruntime}{\econtexRoot/Resources/texmf-local/tex/latex/pdfsuppressruntime}
\providecommand{\econark}{/Volumes/Sync/GitHub/llorracc/SolvingMicroDSOPs/SolvingMicroDSOPs-Latest/.resources/texmf-local/tex/latex/local-econark}
\providecommand{\econtexSetup}{\econtexRoot/Resources/texmf-local/tex/latex/econtexSetup}
\providecommand{\econtexShortcuts}{\econtexRoot/Resources/texmf-local/tex/latex/econtexShortcuts}
\providecommand{\econtexBibMake}{\econtexRoot/Resources/texmf-local/tex/latex/econtexBibMake}
\providecommand{\econtexBibStyle}{\econtexRoot/Resources/texmf-local/bibtex/bst/econtex}
\providecommand{\econtexBib}{economics}
\providecommand{\economics}{\econtexRoot/Resources/texmf-local/bibtex/bib/economics}
\providecommand{\notes}{\econtexRoot/Resources/texmf-local/tex/latex/handout}
\providecommand{\handoutSetup}{\econtexRoot/Resources/texmf-local/tex/latex/handoutSetup}
\providecommand{\handoutShortcuts}{\econtexRoot/Resources/texmf-local/tex/latex/handoutShortcuts}
\providecommand{\handoutBibMake}{\econtexRoot/Resources/texmf-local/tex/latex/handoutBibMake}
\providecommand{\handoutBibStyle}{\econtexRoot/Resources/texmf-local/bibtex/bst/handout}

\providecommand{\FigDir}{\econtexRoot/Figures}
\providecommand{\CodeDir}{\econtexRoot/Code}
\providecommand{\DataDir}{\econtexRoot/Data}
\providecommand{\SlideDir}{\econtexRoot/Slides}
\providecommand{\TableDir}{\econtexRoot/Tables}
\providecommand{\ApndxDir}{\econtexRoot/Appendices}

\providecommand{\ResourcesDir}{\econtexRoot/Resources}
\providecommand{\rootFromOut}{..} % APFach back to root directory from output-directory
\providecommand{\LaTeXGenerated}{\econtexRoot/LaTeX} % Put generated files in subdirectory
\providecommand{\econtexPaths}{\econtexRoot/Resources/econtexPaths}
\providecommand{\LaTeXInputs}{\econtexRoot/Resources/LaTeXInputs}
\providecommand{\LtxDir}{}
\providecommand{\EqDir}{Equations} % Put generated files in subdirectory

\documentclass[\econtexRoot/BufferStockTheory]{subfiles}
\providecommand{\ApndxDir}{}\renewcommand{\ApndxDir}{\econtexRoot/Appendices}
\providecommand{\EqDir}{}\renewcommand{\EqDir}{\econtexRoot/Equations}
\providecommand{\TableDir}{}\renewcommand{\TableDir}{\econtexRoot/Tables}
\providecommand{\FigDir}{}\renewcommand{\FigDir}{\econtexRoot/Figures}
\providecommand{\LaTeXInputs}{}\renewcommand{\LaTeXInputs}{\econtexRoot/@resources/texlive/texmf-local/tex/latex}
\providecommand{\LaTeXGenerated}{}\renewcommand{\LaTeXGenerated}{\econtexRoot} % not worth trying to put generated files in a subdir
\providecommand{\ResourcesDir}{}\renewcommand{\ResourcesDir}{\econtexRoot/@resources}
\providecommand{\LtxDir}{}\renewcommand{\LtxDir}{}
\usepackage{econark-ifsubfile}\usepackage{econark-xrsetup}
\notinsubfile{\externaldocument{\LaTeXGenerated/BufferStockTheory}\providecommand{\texname}{}\renewcommand{\texname}{Introduction}}
\begin{document}
\hypertarget{ApndxCGrowthDeclines}{}
\section{When Is Consumption Growth Declining in \texorpdfstring{$m$}{m}?}\label{sec:ApndxCGrowthDeclines}\label{subsec:dcgdxneg}

Figure~\ref{fig:cNrmTargetFig} depicts the expected consumption growth factor as a strictly
declining function of the cash-on-hand ratio. To investigate this,
define
\begin{align*}
  \pmb{\Upsilon}(\mNrm_{t})  & \equiv  \PermGroFac_{t+1} \usual{\cFunc}(\RNrmByGRnd_{t+1}\aFunc(\mNrm_{t})+\tranShkAll_{t+1})/\usual{\cFunc}(\mNrm_{t})  = \cLvl_{t+1}/\cLvl_{t}
\end{align*}
and the proposition in which we are interested is
\begin{align*}
  (d/d\mNrm_{t})\Ex_{t}[\underbrace{\pmb{\Upsilon}(\mNrm_{t})}_{\equiv \pmb{\Upsilon}_{t+1}}]  & < 0  
\end{align*}
or differentiating through the expectations operator, what we want is
\begin{align}
  \Ex_{t}\left[\PermGroFac_{t+1} \left(\frac{\usual{\cFunc}^{\prime}(\mNrm_{t+1})\RNrmByGRnd_{t+1}\aFunc^{\prime}(\mNrm_{t})\usual{\cFunc}(\mNrm_{t})-\usual{\cFunc}(\mNrm_{t+1})\usual{\cFunc}^{\prime}(\mNrm_{t})}{\usual{\cFunc}{(\mNrm_{t})}^{2}}\right)\right]  & < 0 \label{eq:kappaPrimeLT0}.
\end{align}

Henceforth indicating appropriate arguments by the corresponding
subscript (e.g.\ $\cFunc_{t+1}^{\prime} \equiv \cFunc^{\prime}(\mNrm_{t+1})$), since
$\PermGroFac_{t+1}\RNrmByGRnd_{t+1}=\Rfree$, the portion of the LHS of equation~\eqref{eq:kappaPrimeLT0} in brackets can be manipulated to yield
\begin{equation}\begin{gathered}\begin{aligned}
  \cFunc_{t} \pmb{\Upsilon}^{\prime}_{t+1}  & = \cFunc^{\prime}_{t+1}\aFunc^{\prime}_{t}\Rfree-\cFunc^{\prime}_{t} \PermGroFac_{t+1} \cFunc_{t+1}/\cFunc_{t} \nonumber
  \\  & = \cFunc^{\prime}_{t+1}\aFunc^{\prime}_{t}\Rfree-\cFunc^{\prime}_{t} \pmb{\Upsilon}_{t+1} \label{eq:cPrimek}
        .
\end{aligned}\end{gathered}\end{equation}

Now differentiate the Euler equation with respect to $\mNrm_{t}$:
\begin{equation}\label{eq:covgen}\begin{aligned}
  1  & = \Rfree \DiscFac \Ex_{t}[ \pmb{\Upsilon}_{t+1}^{-\CRRA}] 
  \\ 0  & = \Ex_{t}[\pmb{\Upsilon}_{t+1}^{-\CRRA-1} \pmb{\Upsilon}_{t+1}^{\prime}] 
  \\  & = \Ex_{t}[\pmb{\Upsilon}_{t+1}^{-\CRRA-1}]\Ex_{t}[\pmb{\Upsilon}_{t+1}^{\prime}]+\mbox{cov}_{t}(\pmb{\Upsilon}_{t+1}^{-\CRRA-1},\pmb{\Upsilon}_{t+1}^{\prime}) 
  \\ \Ex_{t}[\pmb{\Upsilon}_{t+1}^{\prime}]  & = -\mbox{cov}_{t}(\pmb{\Upsilon}_{t+1}^{-\CRRA-1},\pmb{\Upsilon}_{t+1}^{\prime})/\Ex_{t}[\pmb{\Upsilon}_{t+1}^{-\CRRA-1}] 
\end{aligned}\end{equation}
but since $\pmb{\Upsilon}_{t+1} > 0$ we can see from~\eqref{eq:covgen} that~\eqref{eq:kappaPrimeLT0} is equivalent to
\begin{equation}\begin{gathered}\begin{aligned}
  \mbox{cov}_{t}(\pmb{\Upsilon}_{t+1}^{-\CRRA-1},\pmb{\Upsilon}_{t+1}^{\prime})  & > 0 \nonumber
\end{aligned}\end{gathered}\end{equation}
which, using~\eqref{eq:cPrimek}, will be true if
\begin{equation}\begin{gathered}\begin{aligned}
  \mbox{cov}_{t}(\pmb{\Upsilon}_{t+1}^{-\CRRA-1},\cFunc^{\prime}_{t+1}\aFunc^{\prime}_{t}\Rfree - \cFunc^{\prime}_{t}\pmb{\Upsilon}_{t+1})  & > 0 \notag
\end{aligned}\end{gathered}\end{equation}
which in turn will be true if both
\begin{equation}\begin{gathered}\begin{aligned}
  \mbox{cov}_{t}(\pmb{\Upsilon}_{t+1}^{-\CRRA-1},\cFunc^{\prime}_{t+1} )  & > 0 \notag
\end{aligned}\end{gathered}\end{equation}
and
\begin{align*}
  \mbox{cov}_{t}(\pmb{\Upsilon}_{t+1}^{-\CRRA-1},\pmb{\Upsilon}_{t+1})  & < 0. \notag
\end{align*}

The latter proposition is obviously true under our assumption $\CRRA > 1$.
The former will be true if
\begin{align*}
  \mbox{cov}_{t}\left({(\PermGroFac \permShk_{t+1} \cFunc(\mNrm_{t+1}))}^{-\CRRA-1},\cFunc^{\prime}(\mNrm_{t+1}) \right)  & > 0 \nonumber.
\end{align*}

The two shocks cause two kinds of variation in $\mNrm_{t+1}$.
Variations due to $\tranShkAll_{t+1}$ satisfy the proposition, since a higher draw of $\tranShkAll$ both reduces $c_{t+1}^{-\CRRA-1}$ and reduces the marginal propensity to consume.
However, permanent shocks have conflicting effects.
On the one hand, a higher draw of $\permShk_{t+1}$ will reduce $\mNrm_{t+1}$, thus increasing both $c_{t+1}^{-\CRRA-1}$ and $c_{t+1}^{\prime}$.
On the other hand, the $c_{t+1}^{-\CRRA-1}$ term is multiplied by $\PermGroFac \permShk_{t+1}$, so the effect of a higher $\permShk_{t+1}$ could be to decrease the first term in the covariance, leading to a negative covariance with the second term.
(Analogously, a lower permanent shock $\permShk_{t+1}$ can also lead a negative correlation.)

\end{document}
\endinput

% Local Variables:
% eval: (setq TeX-command-list  (assq-delete-all (car (assoc "BibTeX" TeX-command-list)) TeX-command-list))
% eval: (setq TeX-command-list  (assq-delete-all (car (assoc "BibTeX" TeX-command-list)) TeX-command-list))
% eval: (setq TeX-command-list  (assq-delete-all (car (assoc "BibTeX" TeX-command-list)) TeX-command-list))
% eval: (setq TeX-command-list  (assq-delete-all (car (assoc "BibTeX" TeX-command-list)) TeX-command-list))
% eval: (setq TeX-command-list  (assq-delete-all (car (assoc "Biber"  TeX-command-list)) TeX-command-list))
% eval: (add-to-list 'TeX-command-list '("BibTeX" "bibtex.%s" TeX-run-BibTeX nil t                                                                              :help "Run BibTeX") t)
% eval: (add-to-list 'TeX-command-list '("BibTeX" "bibtex.%s" TeX-run-BibTeX nil (plain-tex-mode latex-mode doctex-mode ams-tex-mode texinfo-mode context-mode) :help "Run BibTeX") t)
% TeX-PDF-mode: t
% TeX-file-line-error: t
% TeX-debug-warnings: t
% LaTeX-command-style: (("" "%(PDF)%(latex) %(file-line-error) %(extraopts) -output-directory=. %S%(PDFout)"))
% TeX-source-correlate-mode: t
% TeX-parse-self: t
% eval: (cond ((string-equal system-type "darwin") (progn (setq TeX-view-program-list '(("Skim" "/Applications/Skim.app/Contents/SharedSupport/displayline -b %n %o %b"))))))
% eval: (cond ((string-equal system-type "gnu/linux") (progn (setq TeX-view-program-list '(("Evince" "evince --page-index=%(outpage).%o"))))))
% eval: (cond ((string-equal system-type "gnu/linux") (progn (setq TeX-view-program-selection '((output-pdf "Evince"))))))
% TeX-parse-all-errors: t
% End:
