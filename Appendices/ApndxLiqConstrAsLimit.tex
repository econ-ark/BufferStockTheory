% -*- mode: LaTeX; TeX-PDF-mode: t; -*-
% Add the listed directories to the search path
% (allows easy moving of files around later)
% these paths are searched AFTER local config kpsewhich

% *.sty, *.cls
\makeatletter
\def\input@path{{@resources/texlive/texmf-local/tex/latex//}
        ,{@resources/texlive/latex//}
        ,{@local//}
        }
\makeatother
\makeatletter
\def\bibinput@path{{@resources/texlive/texmf-local/tex/latex//}
        ,{@resources/texlive/latex//},
        ,{@local//}
        }
\makeatother
  % allow latex to find custom stuff
% LaTeX path to the root directory of the current project, from the directory in which this file resides
% and path to econtexPaths which defines the rest of the paths like \FigDir
\providecommand{\econtexRoot}{}\renewcommand{\econtexRoot}{.}
\providecommand{\econtexPaths}{}\renewcommand{\econtexPaths}{\econtexRoot/Resources/econtexPaths}
% The \commands below are required to allow sharing of the same base code via Github between TeXLive on a local machine and Overleaf (which is a proxy for "a standard distribution of LaTeX").  This is an ugly solution to the requirement that custom LaTeX packages be accessible, and that Overleaf prohibits symbolic links

\providecommand{\econtex}{\econtexRoot/Resources/texmf-local/tex/latex/econtex}
\providecommand{\pdfsuppressruntime}{\econtexRoot/Resources/texmf-local/tex/latex/pdfsuppressruntime}
\providecommand{\econark}{/Volumes/Sync/GitHub/llorracc/SolvingMicroDSOPs/SolvingMicroDSOPs-Latest/.resources/texmf-local/tex/latex/local-econark}
\providecommand{\econtexSetup}{\econtexRoot/Resources/texmf-local/tex/latex/econtexSetup}
\providecommand{\econtexShortcuts}{\econtexRoot/Resources/texmf-local/tex/latex/econtexShortcuts}
\providecommand{\econtexBibMake}{\econtexRoot/Resources/texmf-local/tex/latex/econtexBibMake}
\providecommand{\econtexBibStyle}{\econtexRoot/Resources/texmf-local/bibtex/bst/econtex}
\providecommand{\econtexBib}{economics}
\providecommand{\economics}{\econtexRoot/Resources/texmf-local/bibtex/bib/economics}
\providecommand{\notes}{\econtexRoot/Resources/texmf-local/tex/latex/handout}
\providecommand{\handoutSetup}{\econtexRoot/Resources/texmf-local/tex/latex/handoutSetup}
\providecommand{\handoutShortcuts}{\econtexRoot/Resources/texmf-local/tex/latex/handoutShortcuts}
\providecommand{\handoutBibMake}{\econtexRoot/Resources/texmf-local/tex/latex/handoutBibMake}
\providecommand{\handoutBibStyle}{\econtexRoot/Resources/texmf-local/bibtex/bst/handout}

\providecommand{\FigDir}{\econtexRoot/Figures}
\providecommand{\CodeDir}{\econtexRoot/Code}
\providecommand{\DataDir}{\econtexRoot/Data}
\providecommand{\SlideDir}{\econtexRoot/Slides}
\providecommand{\TableDir}{\econtexRoot/Tables}
\providecommand{\ApndxDir}{\econtexRoot/Appendices}

\providecommand{\ResourcesDir}{\econtexRoot/Resources}
\providecommand{\rootFromOut}{..} % APFach back to root directory from output-directory
\providecommand{\LaTeXGenerated}{\econtexRoot/LaTeX} % Put generated files in subdirectory
\providecommand{\econtexPaths}{\econtexRoot/Resources/econtexPaths}
\providecommand{\LaTeXInputs}{\econtexRoot/Resources/LaTeXInputs}
\providecommand{\LtxDir}{}
\providecommand{\EqDir}{Equations} % Put generated files in subdirectory

\documentclass[\econtexRoot/BufferStockTheory]{subfiles}

\newcommand{\subname}{ApndxConcaveCFunc}
\providecommand{\ApndxDir}{}\renewcommand{\ApndxDir}{\econtexRoot/Appendices}
\providecommand{\EqDir}{}\renewcommand{\EqDir}{\econtexRoot/Equations}
\providecommand{\TableDir}{}\renewcommand{\TableDir}{\econtexRoot/Tables}
\providecommand{\FigDir}{}\renewcommand{\FigDir}{\econtexRoot/Figures}
\providecommand{\LaTeXInputs}{}\renewcommand{\LaTeXInputs}{\econtexRoot/@resources/texlive/texmf-local/tex/latex}
\providecommand{\LaTeXGenerated}{}\renewcommand{\LaTeXGenerated}{\econtexRoot} % not worth trying to put generated files in a subdir
\providecommand{\ResourcesDir}{}\renewcommand{\ResourcesDir}{\econtexRoot/@resources}
\providecommand{\LtxDir}{}\renewcommand{\LtxDir}{}
 % get directory macros
\usepackage{econark-ifsubfile}        % allow conditional execution of code
\usepackage{econark-xrsetup}          % Xternal crossReferences (from main document)

\xrsetup{\econtexRoot/\texname}


\begin{document}

\section{The Liquidity Constrained Solution as a Limit}\label{sec:LiqConstrAsLimit}

Formally, suppose we change the description of the problem by making
the following two assumptions:
\begin{eqnarray*}
  \pZero   & = 0
  \\  c_{t} & \leq  \mNrm_{t} \label{eq:liqconstr},
\end{eqnarray*}
and we designate the solution to this consumer's problem $\cnstr{\cFunc}_{t}(\mNrm)$.
We will henceforth refer to this as the problem of the `restrained' consumer (and, to avoid a common confusion, we will refer to the consumer as `constrained' only in circumstances when the constraint is actually binding).

Redesignate the consumption function that emerges from our original problem for a given fixed $\pZero$ as $\cFunc_{t}(\mNrm;\pZero)$ where we separate the arguments by a semicolon to distinguish between $\mNrm$, which is a state variable, and $\pZero$, which is not.
The proposition we wish to demonstrate is
\begin{equation}\begin{gathered}\begin{aligned}
      \lim_{\pZero \downarrow 0} \cFunc_{t}(\mNrm;\pZero)  & = \cnstr{\cFunc}_{t}(\mNrm). \label{eq:RestrEqUnrestr} 
    \end{aligned}\end{gathered}\end{equation}

We will first examine the problem in period $T-1$, then argue that the desired result propagates to earlier periods.
For simplicity, suppose that the interest, growth, and time-preference factors are $\DiscFac = \Rfree = \PermGroFac = 1$, and there are no permanent shocks, $\permShk=1$; the results below are easily generalized to the full-fledged version of the problem.

The solution to the restrained consumer's optimization problem can be obtained as follows.
Assuming that the consumer's behavior in period $T$ is given by $\cFunc_{T}(\mNrm)$ (in practice, this will be $\cFunc_{T}(\mNrm)=m$), consider the unrestrained optimization problem
\begin{equation}\begin{gathered}\begin{aligned}
      \cnstr{\aFunc}^{*}_{T-1}(\mNrm)  & = \underset{\aNrm}{\arg \max} \left\{\uFunc(\mNrm-\aNrm) +  \int_{\underline{\tranShkEmp}}^{\bar{\tranShkEmp}} \vFunc_{T}(a+\tranShkEmp) d\CDF_{\tranShkEmp} \right\}. \label{eq:vUnconstr}
    \end{aligned}\end{gathered}\end{equation}

As usual, the envelope theorem tells us that $\vFunc_{T}^{\prime}(\mNrm)=\uP(\cFunc_{T}(\mNrm))$ so the expected marginal value of ending period $T-1$ with assets $\aNrm$ can be defined as
\begin{equation}\begin{gathered}\begin{aligned}
      \cnstr{\mathfrak{v}}_{T-1}^{\prime}(a)  & \equiv  \int_{\underline{\tranShkEmp}}^{\bar{\tranShkEmp}} \uP(\cFunc_{T}(a+\tranShkEmp)) d\CDF_{\tranShkEmp}, \notag
    \end{aligned}\end{gathered}\end{equation}
and the solution to~\eqref{eq:vUnconstr} will satisfy
\begin{equation}\begin{gathered}\begin{aligned}
      \uP(\mNrm-\aNrm)  & =  \cnstr{\mathfrak{v}}_{T-1}^{\prime}(a) \label{eq:uPConstr}.
      % 
    \end{aligned}\end{gathered}\end{equation}

$\cnstr{\aFunc}_{T-1}^{*}(\mNrm)$ therefore answers the question ``With what level of assets would the restrained consumer like to end period $T-1$ if the constraint $c_{T-1} \leq \mNrm_{T-1}$ did not exist?''  (Note that the restrained consumer's income process remains different from the process for the unrestrained consumer so long as $\pZero>0$.)
The restrained consumer's actual asset position will be
\begin{equation}\begin{gathered}\begin{aligned}
      \cnstr{\aFunc}_{T-1}(\mNrm)  & = \max[0,\cnstr{\aFunc}^{*}_{T-1}(\mNrm)], \notag
    \end{aligned}\end{gathered}\end{equation}
reflecting the inability of the restrained consumer to spend more than current resources, and note (as pointed out by \cite{deatonLiqConstr}) that
\begin{equation}\begin{gathered}\begin{aligned}
      \mNrm^{1}_{\#}  & = {\left( \cnstr{\mathfrak{v}}_{T-1}^{\prime}(0)\right)}^{-1/\CRRA} \notag
    \end{aligned}\end{gathered}\end{equation}
is the cusp value of $\mNrm$ at which the constraint makes the
transition between binding and non-binding in period $T-1$.

Analogously to~\eqref{eq:uPConstr}, defining
\begin{equation}\begin{gathered}\begin{aligned}
      \mathfrak{v}_{T-1}^{\prime}(a;\pZero)  & \equiv  \left[\pZero \aNrm^{-\CRRA}+\pNotZero\int_{\underline{\tranShkEmp}}^{\bar{\tranShkEmp}} {\left(\cFunc_{T}(a+\tranShkEmp/\pNotZero)\right)}^{-\CRRA} d\CDF_{\tranShkEmp}\right], \label{eq:vFrakPrime}
    \end{aligned}\end{gathered}\end{equation}
the Euler equation for the original consumer's problem implies
\begin{equation}\begin{gathered}\begin{aligned}
      {(\mNrm-\aNrm)}^{-\CRRA}  & = \mathfrak{v}_{T-1}^{\prime}(a;\pZero) \label{eq:uPUnconstr}
    \end{aligned}\end{gathered}\end{equation}
with solution $\aFunc_{T-1}^{*}(\mNrm;\pZero)$.
Now note that for any fixed $\aNrm>0$, $\lim_{\pZero \downarrow 0} \mathfrak{v}_{T-1}^{\prime}(a;\pZero) = \cnstr{\mathfrak{v}}_{T-1}^{\prime}(a)$.
Since the LHS of~\eqref{eq:uPConstr} and~\eqref{eq:uPUnconstr} are identical, this means that $\lim_{\pZero \downarrow 0} \aFunc_{T-1}^{*}(\mNrm;\pZero) = \cnstr{\aFunc}_{T-1}^{*}(\mNrm)$.
That is, for any fixed value of $\mNrm > \mNrm^{1}_{\#}$ such that the consumer subject to the restraint would voluntarily choose to end the period with positive assets, the level of end-of-period assets for the unrestrained consumer approaches the level for the restrained consumer as $\pZero \downarrow 0$.
With the same $\aNrm$ and the same $\mNrm$, the consumers must have the same $c$, so the consumption functions are identical in the limit.

Now consider values $\mNrm\leq \mNrm^{1}_{\#}$ for which the restrained consumer is constrained.
It is obvious that the baseline consumer will never choose $\aNrm \leq 0$ because the first term in~\eqref{eq:vFrakPrime} is $\lim_{\aNrm \downarrow 0} \pZero \aNrm^{-\CRRA} = \infty$, while $\lim_{\aNrm \downarrow 0} {(\mNrm-\aNrm)}^{-\CRRA}$ is finite (the marginal value of end-of-period assets approaches infinity as assets approach zero, but the marginal utility of consumption has a finite limit for $\mNrm>0$).
The subtler question is whether it is possible to rule out strictly positive $\aNrm$ for the unrestrained consumer.

The answer is yes.
Suppose, for some $\mNrm < \mNrm^{1}_{\#}$, that the unrestrained consumer is considering ending the period with any positive amount of assets $\aNrm=\delta > 0$.
For any such $\delta$ we have that $\lim_{\pZero \downarrow 0} \mathfrak{v}_{T-1}^{\prime}(a;\pZero)=\cnstr{\mathfrak{v}}_{T-1}^{\prime}(a)$.
But by assumption we are considering a set of circumstances in which $\cnstr{\aFunc}_{T-1}^{*}(\mNrm) < 0$, and we showed earlier that $\lim_{\pZero \downarrow 0} \aFunc_{T-1}^{*}(\mNrm;\pZero) = \cnstr{\aFunc}_{T-1}^{*}(\mNrm)$.
So, having assumed $\aNrm = \delta > 0$, we have proven that the consumer would optimally choose $\aNrm < 0$, which is a contradiction.
A similar argument holds for $\mNrm = \mNrm^{1}_{\#}$.

These arguments demonstrate that for any $\mNrm>0$, $\lim_{\pZero \downarrow 0} \cFunc_{T-1}(\mNrm;\pZero) = \cnstr{\cFunc}_{T-1}(\mNrm)$ which is the period $T-1$ version of~\eqref{eq:RestrEqUnrestr}.
But given equality of the period $T-1$ consumption functions, backwards recursion of the same arguments demonstrates that the limiting consumption functions in previous periods are also identical to the constrained function.

Note finally that another intuitive confirmation of the equivalence between the two problems is that our formula~\eqref{eq:MPCmaxdefn} for the maximal marginal propensity to consume satisfies
\begin{eqnarray*}
  \lim_{\pZero \downarrow 0} \MPCmax  & = 1,
\end{eqnarray*}
which makes sense because the marginal propensity to consume for a constrained restrained consumer is 1 by our definitions of `constrained' and `restrained.'

\ifSubfilesClassLoaded{\bibfilesfind{\texname}\bibliography{\bibfilesfound}}{}

\end{document}
\endinput

