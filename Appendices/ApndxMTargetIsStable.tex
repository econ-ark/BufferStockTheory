% -*- mode: LaTeX; TeX-PDF-mode: t; -*-
% LaTeX path to the root directory of the current project, from the directory in which this file resides
% and path to econtexPaths which defines the rest of the paths like \FigDir
\providecommand{\econtexRoot}{}\renewcommand{\econtexRoot}{.}
\providecommand{\econtexPaths}{}\renewcommand{\econtexPaths}{\econtexRoot/Resources/econtexPaths}
% The \commands below are required to allow sharing of the same base code via Github between TeXLive on a local machine and Overleaf (which is a proxy for "a standard distribution of LaTeX").  This is an ugly solution to the requirement that custom LaTeX packages be accessible, and that Overleaf prohibits symbolic links

\providecommand{\econtex}{\econtexRoot/Resources/texmf-local/tex/latex/econtex}
\providecommand{\pdfsuppressruntime}{\econtexRoot/Resources/texmf-local/tex/latex/pdfsuppressruntime}
\providecommand{\econark}{/Volumes/Sync/GitHub/llorracc/SolvingMicroDSOPs/SolvingMicroDSOPs-Latest/.resources/texmf-local/tex/latex/local-econark}
\providecommand{\econtexSetup}{\econtexRoot/Resources/texmf-local/tex/latex/econtexSetup}
\providecommand{\econtexShortcuts}{\econtexRoot/Resources/texmf-local/tex/latex/econtexShortcuts}
\providecommand{\econtexBibMake}{\econtexRoot/Resources/texmf-local/tex/latex/econtexBibMake}
\providecommand{\econtexBibStyle}{\econtexRoot/Resources/texmf-local/bibtex/bst/econtex}
\providecommand{\econtexBib}{economics}
\providecommand{\economics}{\econtexRoot/Resources/texmf-local/bibtex/bib/economics}
\providecommand{\notes}{\econtexRoot/Resources/texmf-local/tex/latex/handout}
\providecommand{\handoutSetup}{\econtexRoot/Resources/texmf-local/tex/latex/handoutSetup}
\providecommand{\handoutShortcuts}{\econtexRoot/Resources/texmf-local/tex/latex/handoutShortcuts}
\providecommand{\handoutBibMake}{\econtexRoot/Resources/texmf-local/tex/latex/handoutBibMake}
\providecommand{\handoutBibStyle}{\econtexRoot/Resources/texmf-local/bibtex/bst/handout}

\providecommand{\FigDir}{\econtexRoot/Figures}
\providecommand{\CodeDir}{\econtexRoot/Code}
\providecommand{\DataDir}{\econtexRoot/Data}
\providecommand{\SlideDir}{\econtexRoot/Slides}
\providecommand{\TableDir}{\econtexRoot/Tables}
\providecommand{\ApndxDir}{\econtexRoot/Appendices}

\providecommand{\ResourcesDir}{\econtexRoot/Resources}
\providecommand{\rootFromOut}{..} % APFach back to root directory from output-directory
\providecommand{\LaTeXGenerated}{\econtexRoot/LaTeX} % Put generated files in subdirectory
\providecommand{\econtexPaths}{\econtexRoot/Resources/econtexPaths}
\providecommand{\LaTeXInputs}{\econtexRoot/Resources/LaTeXInputs}
\providecommand{\LtxDir}{}
\providecommand{\EqDir}{Equations} % Put generated files in subdirectory

\documentclass[\econtexRoot/BufferStockTheory]{subfiles}
%\providecommand{\ApndxDir}{}\renewcommand{\ApndxDir}{\econtexRoot/Appendices}
\providecommand{\EqDir}{}\renewcommand{\EqDir}{\econtexRoot/Equations}
\providecommand{\TableDir}{}\renewcommand{\TableDir}{\econtexRoot/Tables}
\providecommand{\FigDir}{}\renewcommand{\FigDir}{\econtexRoot/Figures}
\providecommand{\LaTeXInputs}{}\renewcommand{\LaTeXInputs}{\econtexRoot/@resources/texlive/texmf-local/tex/latex}
\providecommand{\LaTeXGenerated}{}\renewcommand{\LaTeXGenerated}{\econtexRoot} % not worth trying to put generated files in a subdir
\providecommand{\ResourcesDir}{}\renewcommand{\ResourcesDir}{\econtexRoot/@resources}
\providecommand{\LtxDir}{}\renewcommand{\LtxDir}{}
\usepackage{econark-ifsubfile}\usepackage{econark-xrsetup}
\compilingasstandalone{\externaldocument{BufferStockTheory}\providecommand{\texname}{}\renewcommand{\texname}{Introduction}} % Get xrefs -- esp to appendix -- from main file; only works properly if main file has already been compiled;

%\renewcommand{\LtxDir}{}
\notinsubfile{\renewcommand{\LtxDir}{}}

%\renewcommand\LineNumber{\the\inputlineno}

\begin{document}
%{\linenumbers}

\hypertarget{ApndxMTargetIsStable}{}
\section{Appendix for Section \ref{sec:individStability}}\label{sec:ApndxMTargetIsStable}

\subsection{Asymptotic Consumption Growth Factors}\label{subsec:AppxCgrowthFac}

\begin{proof}[\textbf{Proof for Proposition \ref{prop:convgGrowth}}]

For consumption growth, as $\mNrm \rightarrow 0$ we have:
%
\begin{align}\label{eq:consGrowth}
  \lim_{\mNrm_{t} \rightarrow 0} \Ex_{t}\left[\left(\frac{\cFunc({\mNrm}_{t+1})}{\cFunc(\mNrm_t)}\right){\PermGroFacRnd}_{t+1}\right]
  & > \lim_{\mNrm_{t} \rightarrow 0} \Ex_{t}\left[\left(\frac{\Min{\cFunc}(\RNrmByGRnd_{t+1}\aFunc(\mNrm_{t})+{%
    \tranShkAll}_{t+1})}{\MPCmax \mNrm_{t}}\right){\PermGroFacRnd}_{t+1}\right]  \notag \\
  & = \pZero \lim_{\mNrm_{t} \rightarrow 0} \Ex_{t}\left[\left(\frac{\Min{\cFunc}(\RNrmByGRnd_{t+1}\aFunc(\mNrm_{t}))}{\MPCmax \mNrm_{t}}\right){\PermGroFac}_{t+1}\right] \\
  & ~~~~~~ + \pNotZero \lim_{\mNrm_{t} \rightarrow 0}  \Ex_{t}\left[\left(\frac{\Min{\cFunc}(\RNrmByGRnd_{t+1}\aFunc(\mNrm_{t})+
    \tranShkEmp_{t+1}/\pNotZero)}{\MPCmax \mNrm_{t}}\right){\PermGroFacRnd}_{t+1}\right]  \\\notag
  & > \pNotZero \lim_{\mNrm_{t} \rightarrow 0} \Ex_{t}\left[\left(\frac{\Min{\cFunc}(
    \tranShkEmp_{t+1}/\pNotZero)}{\MPCmax \mNrm_{t}}\right){\PermGroFacRnd}_{t+1}\right] \\
  & = \infty \nonumber
\end{align}
%
where the second-to-last line follows because  $\lim_{\mNrm_{t} \rightarrow 0} \Ex_{t}\left[\left(\frac{\Min{\cFunc}(\RNrmByGRnd_{t+1}\aFunc(\mNrm_{t}))}{\MPCmax \mNrm_{t}}\right){\PermGroFacRnd}_{t+1}\right]$ is positive, and the last line follows because the minimum possible realization of $\tranShkEmp_{t+1}$ is $\Min{\tranShkEmp}>0$ so the minimum possible value of expected next-period consumption is positive.

Next we establish the limit of the expected consumption growth factor as $\mNrm_{t} \rightarrow \infty$:
%
\begin{align*}
  \lim_{\mNrm_{t} \rightarrow \infty} \Ex_{t}[
  \cLvl_{t+1}/\cLvl_{t}]  & = \lim_{\mNrm_{t} \rightarrow \infty} \Ex_{t}[
                            {\PermGroFacRnd}_{t+1} {\cNrm}_{t+1}/c_{t}].
\end{align*}
%
But
\begin{align*}
  \Ex_{t}[{\PermGroFacRnd}_{t+1} {\Min{\cNrm}}_{t+1}/\bar{\cNrm}_{t}] \leq \Ex_{t}[{\PermGroFacRnd}_{t+1} {\cNrm}_{t+1}/\cNrm_{t}] \leq \Ex_{t}[{\PermGroFacRnd}_{t+1} {\bar{\cNrm}}_{t+1}/\Min{\cNrm}_{t}]
\end{align*}
and
\begin{equation*}  \labelsafe{eq:xttoinfty}
  \lim_{\mNrm_t \rightarrow \infty} \PermGroFacRnd_{t+1}\Min{\cFunc}(\mNrm_{t+1})/\bar{\cFunc}(\mNrm_t) =
  \lim_{\mNrm_{t} \rightarrow \infty} \PermGroFacRnd_{t+1}\bar{\cFunc}(\mNrm_{t+1})/\Min{\cFunc}(\mNrm_t) =
  \lim_{\mNrm_{t} \rightarrow \infty}\PermGroFacRnd_{t+1} \mNrm_{t+1}/\mNrm_t,
\end{equation*}
while (for convenience defining $\aFunc(\mNrm_{t})=\mNrm_{t}-\usual{\cFunc}(\mNrm_{t})$), \hypertarget{xtp1toinfty}{}
\begin{align}  \labelsafe{eq:xtp1toinfty}
  \lim_{\mNrm_{t} \rightarrow \infty} \PermGroFacRnd_{t+1} \mNrm_{t+1}/\mNrm_t  & = \lim_{\mNrm_{t} \rightarrow \infty}
                                                                            \left(\frac{\Rfree \aFunc(\mNrm_t)+{\PermGroFacRnd}_{t+1}\tranShkAll_{t+1}}{\mNrm_t}\right)
  \\  & = {(\Rfree \DiscFac)}^{1/\CRRA} = \APFac \notag
\end{align}
because $\lim\limits_{\mNrm_{t}\rightarrow \infty} \aFunc^{\prime}(\mNrm)=\RPFac$\footnote{$\displaystyle \lim\limits_{\mNrm_{t}\rightarrow \infty} \aFunc(\mNrm_{t})/\mNrm_{t}=1-\lim_{\mNrm_{t}\rightarrow \infty} \usual{\cFunc}(\mNrm_{t})/\mNrm_{t}=1-\lim_{\mNrm_{t}\rightarrow \infty}\usual{\cFunc}^{\prime}(\mNrm_{t})=\RPFac$.} and $\PermGroFacRnd_{t+1}\tranShkAll_{t+1}/\mNrm_{t} \leq (\PermGroFac \bar{\permShk} \bar{\tranShkEmp}/\pNotZero )/\mNrm_{t}$ which goes to zero as $\mNrm_{t}$ goes to infinity.
Hence we have:
%
\begin{equation*}
  {\APFac}  \leq \lim_{\mNrm_{t} \rightarrow \infty} \Ex_{t}[\cLvl_{t+1}/\cLvl_{t}] \leq {\APFac}
\end{equation*}
%
so as cash goes to infinity, consumption growth approaches its value $\APFac$ in the perfect foresight model.
\end{proof}

This appendix proves Theorems~\ref{thm:target}-\ref{thm:MSSBalExists} and:
\compilingassubfile{\setcounter{theorem}{1}}

  \begin{lemma}\labelsafe{lemma:orderingPartOne}
  If $\BalGroFac{\mNrm}$ and $\TargetNrm{\mNrm}$ both exist, then $\BalGroFac{\mNrm} \leq \TargetNrm{\mNrm}$.
  \end{lemma}

  \begin{comment}
  \begin{lemma}\labelsafe{lemma:orderingPartTwo}
  If $\BalGroFac{\mNrm}$ and $\BalGroRte{\mNrm}$ both exist, then $\BalGroFac{\mNrm} \leq \BalGroRte{\mNrm}$.
  \end{lemma}
\end{comment}

\subsection{Existence of Buffer Stock Target}

\subsubsection{Existence of Individual Buffer Stock Target}\label{subsubsec:AppxIndividTarget} %
%  \begin{theorem}\iflabelexists{thm:target}{}{\label{thm:target}} % Don't define it if already defined
    For the nondegenerate solution to the problem defined in Section~\ref{subsec:Setup} when {\FVAC}, {\WRIC}, and {\GICMod} all hold, there exists a unique cash-on-hand-to-permanent-income ratio $\mTrgNrm>0$ such that
    \begin{equation}
      \Ex_t [{\mNrm}_{t+1}/\mNrm_t] = 1 \mbox{~if~} \mNrm_t = \mTrgNrm.
      \iflabelexists{eq:mTarget}{}{\label{eq:mTarget}} % Don't define it if already defined
    \end{equation}
    Moreover, $\mTrgNrm$ is a point of `stability' in the sense that
    \begin{equation}
      \iflabelexists{eq:stability}{}{\label{eq:stability}} % Don't define it if already defined
      \begin{split}
        \forall {\mNrm}_t\in(0,\mTrgNrm),      \,\,& \Ex_t [{\mNrm}_{t+1}] > {\mNrm}_t  \\
        \forall {\mNrm}_t\in(\mTrgNrm,\infty), \,\,& \Ex_t [{\mNrm}_{t+1}] < {\mNrm}_t.
      \end{split}
    \end{equation}
  \end{theorem}

  
\begin{proof}[\textbf{Proof of Theorem~\ref{thm:target}}]

First, observe that $\Ex_{t}[{\mNrm}_{t+1}/\mNrm_t] = \frac{\Ex_{t}\left((\mNrm_t - \cFunc(\mNrm_t)) \RNrmByGRnd_{t+1} + \tranShkAll_{t+1}\right)}{\mNrm_t}$.
Note that $\cFunc$ is continuous since $\cFunc$ is concave on $\Reals_{++}$ by Lemma \ref{lemma:MPCBoundsConvg}.
Thus for any convergent sequence $\left\{\mNrm_t^{j}\right\}_{j=0}^{\infty}$, with $\mNrm_t^{j}\in \Reals_{++}$, $(\mNrm_t^{j} - \cFunc(\mNrm_t^{j})) \RNrmByGRnd_{t+1} + \tranShkAll_{t+1}$ will be bounded above and below.
It follows that, using the Dominated Convergence Theorem, $\Ex_{t}[{\mNrm}_{t+1}/\mNrm_t]$ will be continuous in $\mNrm_t$.



The remainder of the proof proceeds as follows.
To establish Equation \eqref{eq:mTarget}, we will show (i) that there exists a point $\breve{\mNrm}_{t}$ where $\Ex_t [\breve{\mNrm}_{t+1}^{\star}/\breve{\mNrm}_{t}^{\star}] < 1$ and (ii) a point $\grave{\mNrm}$ where $\Ex_t [\grave{\mNrm}_{t+1}/\grave{\mNrm}_{t}] > 1$.
By continuity of $\Ex[{\mNrm}_{t+1}/\mNrm_t]$ in $\mNrm_t$ and the Intermediate Value Theorem, there will exist $\mTrgNrm$ such that  $\Ex_t [{\mTrgNrm}_{t+1}/\hat{\mNrm}_{t}] = 1$.
In turn, to establish that $\mTrgNrm$ is a point of stability, Equation \eqref{eq:stability}, we will show that (iii) $\Ex_t [{\mNrm}_{t+1}]-\mNrm_{t}$ is decreasing.


\statement{Part (i).
Existence of \texorpdfstring{$\breve{\mNrm}_{t}$}{m{t}}  where \texorpdfstring{$\Ex_t [\breve{\mNrm}_{t+1}/\breve{\mNrm}_t] < 1$}{E[m{t+1}/m{t}] < 1}.}

To proceed, first suppose \hyperlink{RIC}{return impatience} holds and take the steps analogous to those leading to Equation~\eqref{eq:xtp1toinfty} in the proof of proof for Proposition \ref{prop:convgGrowth},  but dropping the $\PermGroFac_{t+1}$ from the RHS:
\begin{samepage}
\begin{align}
  \lim_{\mNrm_{t} \rightarrow \infty} \Ex_{t}[{\mNrm}_{t+1}/\mNrm_{t}]  & =   
                                                                       \lim_{\mNrm_{t} \rightarrow \infty} 
                                                                       \Ex_{t}\left[\frac{\RNrmByGRnd_{t+1}(\mNrm_{t}-\cFunc(\mNrm_{t}))+{\tranShkAll}_{t+1}}{\mNrm_{t}}\right] \notag 
  \\  & = \Ex_{t}[(\Rfree/{\PermGroFacRnd}_{t+1})\RPFac]  \notag
  \\  & = \Ex_{t}[{\APFac}/{\PermGroFacRnd}_{t+1}]  \labelsafe{eq:emgro}
  \\  & < 1, \notag
\end{align}
\end{samepage}

where the inequality follows from \hyperlink{GICMod}{strong growth impatience}.
By continuity of  $\Ex_{t}[{\mNrm}_{t+1}/\mNrm_t]$ in $\mNrm_t$, there exists $\breve{\mNrm}_{t}$ large enough such that $\Ex_t [\breve{\mNrm}_{t+1}/\breve{\mNrm}_t] < 1$.



Next, suppose \hyperlink{RIC}{return impatience} fails.
The fact that $\lim\limits_{\mNrm_{t} \rightarrow \infty} \frac{\cFunc(\mNrm_{t})}{\mNrm_{t}} = 0$ (Lemma \ref{lemma:MPCBoundsConvg}) means the limit of the RHS of~\eqref{eq:emgro} as $\mNrm_{t} \rightarrow \infty$ is $\bar{\RNrmByGRnd}=\Ex_{t}[\RNrmByGRnd_{t+1}]$.
Equations \eqref{eq:GICStrRICfailst1}-\eqref{eq:GICStrRICfailst2} below show that when \hyperlink{GICMod}{strong growth impatience} holds and \hyperlink{RIC}{return impatience} fails $\bar{\RNrmByGRnd} < 1$.

Thus, we have $\lim\limits_{\mNrm \rightarrow \infty} \Ex[\mNrm_{t+1}/\mNrm_{t}] < 1$ whether the \hyperlink{RIC}{return impatience} holds or fails.

\medskip

\statement{Part (ii).
Existence of \texorpdfstring{$\grave{\mNrm}_{t}$}{m} > 1 where \texorpdfstring{$\Ex_t [\grave{\mNrm}_{t+1}/\grave{\mNrm}_t] > 1$}{E[m{t+1}/m{t}] > 1}.}

Analogous to Equation \eqref{eq:consGrowth},  the ratio of $\Ex_{t}[\mNrm_{t+1}]$ to $\mNrm_{t}$ is unbounded above as $\mNrm_{t} \rightarrow 0$ because $\lim\limits_{\mNrm_{t}\rightarrow 0} \Ex[\mNrm_{t+1}] > 0$.
Thus, if $\Ex_t [{\mNrm}_{t+1}/\mNrm_t]$ is continuous in $\mNrm_t$, and takes on values above and below one at $\grave{\mNrm}_{t}$ and $\breve{\mNrm}_{t}$, by the Intermediate Value Theorem, there must be at least one point at which it is equal to one.

\statement{Part (iii).
\texorpdfstring{$\Ex_t [{\mNrm}_{t+1}] -\mNrm_t$}{Delta m} is strictly decreasing.}

Finally to show $\Ex_t [{\mNrm}_{t+1}] -\mNrm_t$ is strictly decreasing $\mNrm_t$, define \providecommand{\difFunc}{\pmb{\zeta}} $\difFunc(\mNrm_t) \colon = 
\Ex_t[\mNrm_{t+1}] - \mNrm_t$ and note that:
%
\begin{align}\labelsafe{eq:difNrmioEquiv}
  \difFunc(\mNrm_t) < 0 &\leftrightarrow \Ex_t[{\mNrm}_{t+1}/\mNrm_t] < 1 
                          \nonumber\\
  \difFunc(\mNrm_t) = 0 &\leftrightarrow \Ex_t[{\mNrm}_{t+1}/\mNrm_t] = 1\\
  \difFunc(\mNrm_t) > 0 &\leftrightarrow \Ex_t[{\mNrm}_{t+1}/\mNrm_t] > 
                          1,\nonumber,
\end{align}
%
so that $\difFunc(\mTrgNrm)=0$.
Our goal is to prove that $\difFunc(\bullet)$ is strictly  decreasing on $(0,\infty)$.
Let $\Delta_{\epsilon}$ be the finite forward difference for spacing $\epsilon>0$.
Fixing $\epsilon>0$, we will have: 
%
\begin{align}
  \Delta_{\epsilon}\difFunc(\mNrm_{t}) & 			= \Ex_{t}\left[\Delta_{\epsilon} \left( {\RNrmByGRnd}(\mNrm_{t}-\cFunc(\mNrm_{t}))+%
                                                                       {\tranShkAll}_{t+1} - {\mNrm}_t\right) \right] \notag \\
                                                  & =  \bar{\RNrmByGRnd}\left( \epsilon-
                                                       \Delta_{\epsilon}\cFunc({\mNrm}_t)\right) - \epsilon = \epsilon\left( \bar{\RNrmByGRnd}\left[1-
                                                       \frac{\Delta_{\epsilon}\cFunc({\mNrm}_t)}{\epsilon}\right]- 1\right). \labelsafe{eq:finiteDiff2}
\end{align}
%
Notice  that $\frac{\Delta_{\epsilon}\cFunc({\mNrm}_t)}{\epsilon} \leq \frac{\cFunc(\mNrm_{t})}{\mNrm_{t}}<1$ since $\frac{\cFunc(\mNrm_{t})}{\mNrm_{t}}$ is decreasing in $\mNrm_{t}$ by Claim \ref{claim:rationondec} in Appendix \ref{sec:realanalysis}.
Consider the case when \hyperlink{RIC}{return impatience} holds.
Equation~\eqref{eq:MPCminDef} and Lemma \ref{lemma:MPCBoundsConvg} indicate $0 < \MPCmin \leq \frac{\cFunc(\mNrm_{t})}{\mNrm_{t}} < 1$.
%
It follows that:
%
\begin{align*}
  \bar{\RNrmByGRnd}\left[1-\frac{\Delta_{\epsilon}\cFunc^{\prime}({\mNrm}_t)}{\epsilon}\right]- 1 & \leq  \bar{\RNrmByGRnd}(1-\underbrace{(1-\RPFac)}_{\MPCmin}) - 1  \\
                                                            & = \bar{\RNrmByGRnd}\RPFac - 1 \\
                                                            & = \Ex_{t}\left[\frac{\Rfree}{\PermGroFac \permShk_{t+1}}\frac{\APFac}{\Rfree}\right] - 1 \\
                                                            & = \underbrace{\Ex_{t}\left[\frac{\APFac}{\PermGroFac \permShk_{t+1}}\right]}_{= \GPFacMod} - 1 
\end{align*}
%
which is negative because the \hyperlink{GICMod}{strong growth impatience} says $\GPFacMod < 1$.
Conversely, when \hyperlink{RIC}{return impatience} holds fails, recall $\lim\limits_{\mNrm_{t} \rightarrow \infty} \frac{\cFunc(\mNrm_{t})}{\mNrm_{t}} = 0$.
This means $ \Delta_{\epsilon}\difFunc(\mNrm_{t})$ from~\eqref{eq:finiteDiff2} is guaranteed to be negative if:
%
\begin{align}
  \bar{\RNrmByGRnd} = \Ex_{t}\left[\frac{\Rfree}{\PermGroFac \permShk_{t+1}}\right] & < 1  \labelsafe{eq:RbarBelowOne}.
\end{align}
%
But the combination of the \hyperlink{GICMod}{strong growth impatience} holding and the \hyperlink{RIC}{return impatience} failing can be written:
%
\begin{align}\label{eq:GICStrRICfailst1}
  \overbrace{\Ex_{t}\left[\frac{\APFac}{\PermGroFac \permShk_{t+1}}\right]}^{\GPFacMod} & < 1 < \overbrace{\frac{\APFac}{\Rfree}}^{{\RPFac}},
\end{align}
%
and multiplying all three elements by $\Rfree/\APFac$ gives:
%
\begin{align}\label{eq:GICStrRICfailst2}
  \Ex_{t}\left[\frac{\Rfree}{\PermGroFac \permShk_{t+1}}\right] & < \Rfree/\APFac < 1,
\end{align}
%
which satisfies our requirement in~\eqref{eq:RbarBelowOne}, thus completing the proof.
\


\end{proof}
  
\subsubsection{Existence of Pseudo-Steady-State}\label{subsubsec:AppxPseudoSS}

\begin{proof}[\textbf{Proof of Theorem~\ref{thm:MSSBalExists}}]

Since by assumption $ 0 < \Min{\permShk} \leq \permShk_{t+1} \leq \bar{\permShk} < \infty$, our proof in~\ref{subsubsec:AppxIndividTarget} that demonstrated existence and continuity of $\Ex[\mNrm_{t+1}/\mNrm_{t}]$ implies existence and continuity of $\Ex[\permShk_{t+1}\mNrm_{t+1}/\mNrm_{t}]$.

\statement{Part (i).
Existence of a stable point}


Since by assumption $ 0 < \Min{\permShk} \leq \permShk_{t+1} \leq \bar{\permShk} < \infty$, our proof in Subsection~\ref{subsubsec:AppxIndividTarget} that the ratio of $\Ex[\mNrm_{t+1}]$ to $\mNrm_{t}$ is unbounded as $\mNrm_{t} \rightarrow 0$ implies that the ratio $\Ex[\permShk_{t+1}\mNrm_{t+1}]$ to $\mNrm_{t}$ is unbounded as $\mNrm_{t} \rightarrow 0$.
The limit of the expected ratio as $\mNrm_{t}\rightarrow \infty$ goes to infinity is can be found as follows:
%
\begin{align}
  \lim_{\mNrm_{t} \rightarrow \infty} \Ex_{t}[\permShk_{t+1}\mNrm_{t+1}/\mNrm_{t}]  & =   
                                                                  \lim_{\mNrm_{t} \rightarrow \infty} 
                                                                  \Ex_{t}\left[\frac{\PermGroFacRnd_{t+1}\left((\Rfree/\PermGroFacRnd_{t+1})\aFunc(\mNrm_{t})+{\tranShkAll}_{t+1}\right)/\PermGroFac}{\mNrm_{t}}\right] \notag 
  \\   & =   \lim_{\mNrm_{t} \rightarrow \infty} \Ex_{t}\left[
         \frac{(\Rfree/\PermGroFac)\aFunc(\mNrm_{t})+\permShk_{t+1}{\tranShkAll}_{t+1}}{\mNrm_{t}}
         \right] \notag 
  \\   & =   \lim_{\mNrm_{t} \rightarrow \infty} \left[
         \frac{(\Rfree/\PermGroFac)\aFunc(\mNrm_{t})+1}{\mNrm_{t}}
         \right] \notag 
  \\  & = (\Rfree/\PermGroFac)\RPFac \labelsafe{eq:emgro2}
  \\  & = \GPFacRaw \notag
  \\  & < 1, \notag
\end{align}

where the last two lines are merely a restatement of \hyperlink{GIC}{growth impatience}.


To conclude Part (i) of the proof, the Intermediate Value Theorem says that if $\Ex[\permShk_{t+1}\mNrm_{t+1}/\mNrm_t]$ is continuous, and takes on values above and below 1, there must be at least one point at which it is equal to one.

\statement{Part (ii).
\texorpdfstring{$\Ex_{t}[\permShk_{t+1}\mNrm_{t+1}] -\mNrm_t$}{PermShk m{t+1} --- m{t}} is monotonically decreasing.}

Define \providecommand{\difFunc}{\pmb{\zeta}} $\difFunc(\mNrm_t) \colon = 
\Ex_t[\permShk_{t+1}\mNrm_{t+1}] - \mNrm_t$ and note that:
\begin{align}\labelsafe{eq:difLvlEquiv}
  \difFunc(\mNrm_t) < 0 &\leftrightarrow \Ex_t[\permShk_{t+1}\mNrm_{t+1}/\mNrm_{t}] < 1 
                          \nonumber\\
  \difFunc(\mNrm_t) = 0 &\leftrightarrow \Ex_t[\permShk_{t+1}\mNrm_{t+1}/\mNrm_{t}] = 1\\
  \difFunc(\mNrm_t) > 0 &\leftrightarrow \Ex_t[\permShk_{t+1}\mNrm_{t+1}/\mNrm_{t}] > 
                          1,\nonumber
\end{align}
so that $\difFunc(\mTrgNrm)=0$.
Our goal is to prove that $\difFunc(\bullet)$ is strictly 
decreasing on $(0,\infty)$.
Letting $\Delta_{\epsilon}$ be the forward difference operator, we have:
%
\begin{align}
  \Delta_{\epsilon}\difFunc(\mNrm_{t}) & 			= \Ex\left[
                                                                                              \Delta_{\epsilon} \left( 
                                                                                               \frac{\Rfree}{\PermGroFac}(\mNrm_{t}-\cFunc(\mNrm_{t}))+%
                                                                                               {\permShk}_{t+1}{\tranShkAll}_{t+1} - {\mNrm}_t\right) \right] \notag \\
                                                                                             & = \frac{\Rfree}{\PermGroFac}\left( \epsilon-
                                                       \Delta_{\epsilon}\cFunc^{\prime}({\mNrm}_t)\right) - \epsilon = \epsilon\left(  \frac{\Rfree}{\PermGroFac}\left[1-
                                                       \frac{\Delta_{\epsilon}\cFunc({\mNrm}_t)}{\epsilon}\right]- 1\right). \labelsafe{eq:finiteDiff}
\end{align}
%
for any given $\epsilon>0$.
Notice  that $\frac{\Delta_{\epsilon}\cFunc^{\prime}({\mNrm}_t)}{\epsilon} \leq \frac{\cFunc(\mNrm_{t})}{\mNrm_{t}}<1$ since $\frac{\cFunc(\mNrm_{t})}{\mNrm_{t}}$ is decreasing in $\mNrm_{t}$ by Claim \ref{claim:rationondec} in Appendix.
Now, we show that $\difFunc(\mNrm)$ is decreasing when \hyperlink{RIC}{return impatience} holds and when \hyperlink{RIC}{return impatience} fails.
When \hyperlink{RIC}{return impatience} holds,  Equation~\eqref{eq:MPCminDef} and Lemma \ref{lemma:MPCBoundsConvg} indicate that $\MPCmin >0$ and $0 < \MPCmin \leq \frac{\cFunc(\mNrm_{t})}{\mNrm_{t}} < 1$.
It follows that:
%
\begin{align*}
   \frac{\Rfree}{\PermGroFac}\left(1-\cFunc^{\prime}({\mNrm}_t)\right) - 1 & <   \frac{\Rfree}{\PermGroFac}(1-\underbrace{(1-\RPFac)}_{\MPCmin}) - 1  \\
                                                      & = (\Rfree/\PermGroFac)\RPFac - 1, 
\end{align*}
%
which is negative because \hyperlink{GIC}{growth impatience} says $\GPFacRaw < 1$.
Conversely, when \hyperlink{RIC}{return impatience} holds fails, recall $\lim\limits_{\mNrm_{t} \rightarrow \infty} \frac{\cFunc(\mNrm_{t})}{\mNrm_{t}} = 0$.
In turn, this means $\Delta_{\epsilon}\difFunc(\mNrm_{t})$ from~\eqref{eq:finiteDiff} is guaranteed to be negative if:
%
\begin{align}
  (\Rfree/\PermGroFac) & < 1  \labelsafe{eq:FHWCFails}.
\end{align}

But we showed in Section~\ref{subsubsec:PFUncon}, Equation \eqref{eq:RICimplies}, that the only circumstances under which the problem has a non-degenerate solution while \hyperlink{RIC}{return impatience} fails were ones where the \hyperlink{FHWC}{finite limiting human wealth} also fails.
Thus, $(\Rfree/\PermGroFac) < 1$, completing the proof.


\end{proof}

\end{document}
